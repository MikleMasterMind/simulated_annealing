\documentclass[openany, twoside, a4paper, 12pt]{extbook}
\usepackage{amsmath}
\usepackage{amsfonts}
\usepackage{geometry}
\usepackage[russian]{babel}
\usepackage{fancyhdr}
\usepackage{titlesec}
\usepackage{indentfirst}
\usepackage{lipsum}
\usepackage{graphicx}

% Настройка страницы
\geometry{a4paper, margin=1in}
\setlength{\headheight}{15pt}
\pagenumbering{gobble}

% Стиль страницы
\fancypagestyle{mystyle}{
    \fancyhf{} % очищаем текущие значения
    \fancyfoot[C]{\thepage} % Номер страницы по центру снизу
    \renewcommand{\headrulewidth}{0pt} % убираем линию
}

\begin{document}

\title{\textbf{Формальная постановка задачи составления расписания с применением алгоритма имитации отжига}}
\author{Ефанов Михаил Михайлович}
\date{\today}
\maketitle

\newpage
\pagenumbering{arabic}
\pagestyle{mystyle}

\section*{Формальная постановка задачи}

\subsection*{Исходные данные:}
\begin{itemize}
    \item $N$ -- общее количество работ;
    \item $J = \{j_1, j_2, \dots, j_N\}$ -- множество всех работ;
    \item $\tau = \{t_1, t_2, \dots, t_N\}$ -- множество времен выполнения соответствующих работ $j_i$, где $\forall i \in \overline{1,N}$ выполняется $t_i > 0$;
    \item $M$ -- количество доступных процессоров;
    \item $P = \{p_1, p_2, \dots, p_M\}$ -- множество процессоров для выполнения работ.
\end{itemize}

\subsection*{Определение расписания:}
Расписание представляется в виде матрицы $S \in \mathbb{Z}^{M \times N}$, где каждый элемент $s_{jk}$ обозначает порядковый номер работы, выполняемой на процессоре $j$ в позиции $k$, при этом:

\begin{itemize}
    \item $j \in \overline{1, M}$ -- индекс процессора
    \item $k \in \overline{1, L_j}$ -- порядковая позиция в очереди процессора $j$
    \item $L_j$ -- количество работ, назначенных на процессор $j$
    \item $s_{jk} \in \overline{1, N} \cup \{0\}$, где 0 означает отсутствие работы в данной позиции
\end{itemize}

\subsection*{Условия корректности расписания:}
\begin{enumerate}
    \item \textbf{Полнота выполнения:} Все работы должны быть назначены:
    \[
    \bigcup_{j=1}^{M} \{s_{jk} \mid k = 1, \dots, L_j,\ s_{jk} \neq 0\} = \{1, 2, \dots, N\}
    \]
    
    \item \textbf{Уникальность назначения:} Каждая работа выполняется ровно на одном процессоре:
    \[
    \forall i \in \overline{1,N} \ \exists! (j,k): s_{jk} = i
    \]
    
    \item \textbf{Непересекаемость выполнения:} На каждом процессоре работы выполняются последовательно без перекрытий:
    \[
    \forall j \in \overline{1,M} \ \forall k_1 \neq k_2: s_{jk_1} \neq s_{jk_2}
    \]
\end{enumerate}

\subsection*{Цель задачи:}
Требуется найти такое расписание $S$, которое минимизирует заданный критерий качества при выполнении всех указанных условий.

\subsection*{Минимизируемый критерий:}
На основе значения $CRC32_{ЕФановММ} = 1077734097$ выбран первый критерий для реализации.

\begin{itemize}
    \item \textbf{Критерий $K_1$} (критерий разбалансированности расписания)
\end{itemize}

\subsubsection*{Формулировка критерия разбалансированности:}
Пусть $G_j = \{s_{jk} \mid k = 1, \dots, L_j,\ s_{jk} \neq 0\}$ -- множество работ, назначенных на $j$-й процессор. Тогда:
\[
T_j = \sum_{i \in G_j} t_i
\]
представляет собой общее время выполнения всех работ на процессоре $j$.

Критерий разбалансированности определяется как:
\[
K_1 = T_{\text{max}} - T_{\text{min}}
\]
где:
\[
T_{\text{max}} = \max_{j \in \overline{1, M}} T_j
\]
\[
T_{\text{min}} = \min_{j \in \overline{1, M}} T_j
\]

\section*{Ограничения модели}

\begin{itemize}
    \item Каждый процессор $p_j \in P$ может одновременно выполнять не более одной работы;
    \item Выполнение работ происходит без прерываний;
    \item Переключение между работами на процессоре происходит мгновенно;
    \item Время выполнения $t_i \in \tau$ является фиксированной величиной;
    \item Работы на каждом процессоре выполняются в порядке, заданном матрицей расписания $S$.
\end{itemize}

\newpage

\end{document}